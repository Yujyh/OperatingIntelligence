\documentclass[11pt]{article}
\usepackage[utf8]{inputenc}
\usepackage{geometry}
\geometry{margin=0.7in}
\usepackage{parskip}
\usepackage{hyperref}
\begin{document}

\begin{center}
  {\LARGE \textbf{Pilot Brief}}\


  {\large Operating Intelligence: Proactive, Human-Centered OS Assistance}\


  Thomas Gere \\
  thomas@agylgroup.com
\end{center}

\hrule
\vspace{8pt}

\textbf{Objective} \\
Run a 8–12 week pilot to evaluate Operating Intelligence (OI) in a real-world context. Focus areas: child tutoring with guardian controls, workplace deep-work and micro-learning, and community upskilling.

\textbf{Key Features to Test} \\
- Context-aware lesson staging and adaptive difficulty for learners. \\
- Deep-work scheduling with break nudges and micro-learning prompts for professionals. \\
- Privacy-preserving detection of skill gaps and just-in-time micro-credentials.

\textbf{Success Metrics} \\
- Education: learning gains (pre/post), time-on-task quality, guardian satisfaction. \\
- Workplace: throughput, error rates, adherence to healthy breaks, training completion. \\
- Equity: performance across connectivity and device constraints.

\textbf{Data and Privacy} \\
Local-first processing; only aggregated, consented telemetry leaves devices. Pilot will use synthetic or consented datasets and provide opt-in consent flows.

\textbf{Timeline and Resources} \\
- Weeks 0–2: setup, consent, baseline measures. \\
- Weeks 3–10: active pilot with iterative UX adjustments. \\
- Weeks 11–12: analysis, report, and recommendations.

\textbf{Partner Ask} \\
- Provide 10–30 participants per pilot cohort. \\
- Provide minimal integration support for wearable or calendar telemetry. \\
- Participate in weekly check-ins and final evaluation.

\textbf{Contact} \\
Thomas Gere — thomas@agylgroup.com

\end{document}
